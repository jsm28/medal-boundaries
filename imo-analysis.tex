% Copyright 2015 Joseph Samuel Myers.
%
% This program is free software; you can redistribute it and/or modify
% it under the terms of the GNU General Public License as published by
% the Free Software Foundation; either version 3 of the License, or
% (at your option) any later version.
%
% This program is distributed in the hope that it will be useful, but
% WITHOUT ANY WARRANTY; without even the implied warranty of
% MERCHANTABILITY or FITNESS FOR A PARTICULAR PURPOSE.  See the GNU
% General Public License for more details.
%
% You should have received a copy of the GNU General Public License
% along with this program.  If not, see
% <https://www.gnu.org/licenses/>.
\documentclass[a4paper,11pt]{article}
\usepackage[a4paper,left=2.5cm,right=2.5cm,bottom=2.5cm,top=2.5cm]{geometry}
\usepackage{hyperref}
\begin{document}
\title{Analysis of IMO medal boundary choices}
\author{Joseph Myers}
\date{December 2015}
\maketitle

\section*{Introduction}

Medal boundaries at the International Mathematical Olympiad are
determined by the following from the General Regulations:

\begin{quote}
\textbf{5.6} The total number of prizes (first, second and third) must
be approved by the Jury and should not normally exceed half the total
number of Contestants. The numbers of first, second and third prizes
must be approximately in the ratio $1:2:3$.
\end{quote}

This is a revision of previous wording used in 2011 and 2012, the
regulations of which IMOs were used as the basis for the General
Regulations:

\begin{quote}
\textbf{5.6} The total number of prizes (first, second and third) will
not exceed half the total number of Contestants. The numbers of first,
second and third prizes will be approximately in the ratio $1:2:3$.
\end{quote}

The principles date back at least as far as IMO 1984 (whose
regulations appear in \textit{International Mathematical Olympiads
  1978--1985 and Forty Supplementary Problems} by Murray S. Klamkin
(MAA)):

\begin{quote}
The total number of awarded prizes will not exceed half of the number
of all contestants.  The number of 1st, 2nd, and 3rd prizes awarded
will if possible be in the ratio $1:2:3$.
\end{quote}

In practice, first, second and third prizes are generally known as
gold, silver and bronze medals.  They are awarded based on ranking
contestants by total score, and as many medals in total are awarded as
possible consistent with not giving more than half of contestants
medals.  On several occasions, more than half of contestants have been
given medals where this seemed fairer to the Jury in light of the
particular distribution of scores at that IMO, leading to the
insertion of ``normally'' to reflect existing practice when the
regulations were split into General and Annual Regulations, with
revisions of the General Regulations being the responsibility of the
Jury instead of being decided by each host country.

At IMO 2015, changes were debated to make it harder to choose to give
more than half the contestants medals, with the result that it was
agreed (via a change to the Annual Regulations approved by the IMO
Advisory Board for 2015 only) that giving more than half the
contestants medals would require a 2/3 majority.  In addition, a new
procedure was introduced where the medal boundaries were debated and
voted on based on figures and bar charts for the numbers of medals of
each type, without information about the scores to which those
corresponded, to make it less likely that votes would be based on
self-interest.  However, the debate about the merits of different
boundaries still continued for a long time.

This document analyses various possible algorithms for deciding medal
boundaries based on how well they agree with past decisions made by
the Jury, with the idea that it would be fairer if a consistent
algorithm were used to decide medal boundaries (whether always, or
unless the Jury decides by a super-majority that the results of the
algorithm are clearly inappropriate in a particular case).

Some other mathematical olympiads also decide medal boundaries
following similar rules to the IMO.

\section*{Bronze medal boundary}

The algorithms here relate to choosing the total number of contestants
to receive a medal.  All these algorithms choose between two
possibilities: the closest numbers above and below half the number of
contestants.  (If it is possible for exactly half the contestants to
receive a medal, there is only one possibility.)

The following table shows the ideal number of medals each year and the
choices for how many medals to go below or above that number, with the
choice made in bold.

\begin{tabular}{cccc}
Year & Ideal & Below & Above \\
\input{gen-bronze-table.tex}
\end{tabular}

The following list shows which years in the past thirty years
(1986--2015) various algorithms would have failed to predict the
bronze medal boundary correctly (the arbitrary nature of some of the
algorithms is because they are chosen to model many past choices,
rather than on any theoretical basis).  Algorithms are described in
terms of a choice between going $a$~medals above half the number of
contestants and going $b$~medals below; after the first two algorithms
listed, they attempt to model various notions of an unusual
distribution of marks that might justify going over half the number of
contestants.

\begin{itemize}
\input{gen-bronze-alg-list.tex}
\end{itemize}

\section*{Gold and silver medal boundaries}

The algorithms here all work based on a previously determined bronze
medal boundary; for the analysis here, this is the boundary actually
chosen by the Jury.  This is to accord most closely with historical
practice, although of course algorithms could be adapted to determine
all three boundaries together in a similar way.

\section*{References}

All the code used to implement different algorithms for the above
analyses is available at:

\begin{quote}
\texttt{git://git.ukmt.org.uk/git/medal-boundaries.git}
\end{quote}

There is also a mirror on GitHub:

\begin{quote}
\texttt{\href{https://github.com/jsm28/medal-boundaries}{https://github.com/jsm28/medal-boundaries}}
\end{quote}

This analysis document may be revised from time to time.  See that
repository for details of previous versions.

The code downloads data from imo-official.  Note that the ideal number
of medals is half the number of non-disqualified contestants (not half
a total number that includes disqualified contestants, in 1991 and
2010), and in 2005 all calculations are done without including two
contestants to whom a translated paper was transmitted inaccurately
(because that was the basis on which boundaries were determined that
year, with those contestants then being given prizes as if they had
scored~7 on the affected problem).

\end{document}
