% Copyright 2015-2018 Joseph Samuel Myers.
%
% This program is free software; you can redistribute it and/or modify
% it under the terms of the GNU General Public License as published by
% the Free Software Foundation; either version 3 of the License, or
% (at your option) any later version.
%
% This program is distributed in the hope that it will be useful, but
% WITHOUT ANY WARRANTY; without even the implied warranty of
% MERCHANTABILITY or FITNESS FOR A PARTICULAR PURPOSE.  See the GNU
% General Public License for more details.
%
% You should have received a copy of the GNU General Public License
% along with this program.  If not, see
% <https://www.gnu.org/licenses/>.
\documentclass[a4paper,11pt]{article}
\usepackage[a4paper,left=2.5cm,right=2.5cm,bottom=2.5cm,top=2.5cm]{geometry}
\usepackage{hyperref}
\begin{document}
\title{Analysis of IMO medal boundary choices}
\author{Joseph Myers}
\date{June 2018}
\maketitle

\section*{Introduction}

Medal boundaries at the International Mathematical Olympiad are
determined by the following from the General Regulations:

\begin{quote}
\textbf{5.6} The total number of prizes (first, second and third) must
be approved by the Jury and must not exceed half the total number of
Contestants unless this is approved by at least two thirds of the
members of the Jury. The numbers of first, second and third prizes
must be approximately in the ratio $1:2:3$.
\end{quote}

This was agreed in 2016 (with effect from 2017, but with the substance
of the changes having been applied in 2015 and 2016 via the Annual
Regulations), replacing the following:

\begin{quote}
\textbf{5.6} The total number of prizes (first, second and third) must
be approved by the Jury and should not normally exceed half the total
number of Contestants. The numbers of first, second and third prizes
must be approximately in the ratio $1:2:3$.
\end{quote}

This was a revision of previous wording used in 2011 and 2012, the
regulations of which IMOs were used as the basis for the General
Regulations:

\begin{quote}
\textbf{5.6} The total number of prizes (first, second and third) will
not exceed half the total number of Contestants. The numbers of first,
second and third prizes will be approximately in the ratio $1:2:3$.
\end{quote}

The principles date back at least as far as IMO 1984 (whose
regulations appear in \textit{International Mathematical Olympiads
  1978--1985 and Forty Supplementary Problems} by Murray S. Klamkin
(MAA)):

\begin{quote}
The total number of awarded prizes will not exceed half of the number
of all contestants.  The number of 1st, 2nd, and 3rd prizes awarded
will if possible be in the ratio $1:2:3$.
\end{quote}

In practice, first, second and third prizes are generally known as
gold, silver and bronze medals.  They are awarded based on ranking
contestants by total score, and as many medals in total are awarded as
possible consistent with not giving more than half of contestants
medals.  On several occasions, before the introduction of the
2/3-majority requirement, more than half of contestants have been
given medals where this seemed fairer to the Jury in light of the
particular distribution of scores at that IMO, leading to the
insertion of ``normally'' to reflect existing practice when the
regulations were split into General and Annual Regulations, with
revisions of the General Regulations being the responsibility of the
Jury instead of being decided by each host country.

At IMO 2015, changes were debated to make it harder to choose to give
more than half the contestants medals, with the result that it was
agreed (via a change to the Annual Regulations approved by the IMO
Advisory Board for 2015 only) that giving more than half the
contestants medals would require a 2/3 majority; this was
applied via the Annual Regulations again in 2016, and was adopted
permanently from 2017.  In addition, a new
procedure was introduced in 2015 where the medal boundaries were debated and
voted on based on figures and bar charts for the numbers of medals of
each type, without information about the scores to which those
corresponded, to make it less likely that votes would be based on
self-interest.  However, the debate about the merits of different
boundaries still continued for a long time.

This document analyses various possible algorithms for deciding medal
boundaries based on how well they agree with past decisions made by
the Jury, with the idea that it would be fairer if a consistent
algorithm were used to decide medal boundaries (whether always, or
unless the Jury decides by a super-majority that the results of the
algorithm are clearly inappropriate in a particular case).  One idea
suggested at IMO 2015, to allow the desired proportions to be attained
more exactly, was to introduce tie-breaking rules between contestants
on the same score (for example, based on the sum of squares of
individual problem scores).  Such suggestions are not covered in this
analysis, but even if used, algorithms like those here would still be
relevant because of students at the boundaries with the same multiset
of scores (or if applicable with the rule in question, the same scores
on every problem).

Some other mathematical olympiads also decide medal boundaries
following similar rules to the IMO.

\section*{Bronze medal boundary}

The algorithms here relate to choosing the total number of contestants
to receive a medal.  All these algorithms choose between two
possibilities: the closest numbers above and below half the number of
contestants.  (If it is possible for exactly half the contestants to
receive a medal, there is only one possibility.)

The following table shows the ideal number of medals each year and the
choices for how many medals to go below or above that number, with the
choice made in bold.

\begin{tabular}{cccc}
Year & Ideal & Below & Above \\
\input{gen-bronze-table.tex}
\end{tabular}

The following list shows which years from 1986 onwards
various algorithms would have failed to predict the
bronze medal boundary correctly (the arbitrary nature of some of the
algorithms is because they are chosen to model many past choices,
rather than on any theoretical basis).  Algorithms are described in
terms of a choice between going $a$~medals above half the number of
contestants and going $b$~medals below; after the first two algorithms
listed, they attempt to model various notions of an unusual
distribution of marks that might justify going over half the number of
contestants.

\begin{itemize}
\input{gen-bronze-alg-list.tex}
\end{itemize}

\section*{Gold and silver medal boundaries}

The algorithms here all work based on a previously determined bronze
medal boundary; for the analysis here, this is the boundary actually
chosen by the Jury.  This is to accord most closely with historical
practice, although of course algorithms could be adapted to determine
all three boundaries together in a similar way.

As there are two boundaries to be chosen here, to approximate a ratio
$1:2:3$, there may not necessarily be only two obvious choices in all
cases.  We consider the following algorithms (in all cases, ties are
broken by being generous---first by being generous regarding the total
number of gold and silver medals, then by being generous regarding the
number of gold medals).  Each algorithm is given a code used in the
following table.

\begin{itemize}
\item Boundaries may be chosen independently, to make the number of
  gold and silver medals as close as possible to half the number of
  medals, and to make the number of gold medals as close as possible
  to one sixth of the number of medals.  ($I$)
\item Or the silver boundary may be chosen first, to make the number
  of gold and silver medals as close as possible to half the number of
  medals, followed by choosing the gold boundary to make the number of
  gold medals as close as possible to a third of the number of gold
  and silver medals.  ($S$)
\item Or we may try to minimise the total deviation of the numbers of
  each type of medal from the ideal by looking at the $L^p$ norm of
  the vector of deviations from the ideal; or we may do this after
  scaling the numbers of each type of medal so that all the ideal
  numbers are equal.  ($L^p$, $L^ps$)
\item Finally, we consider minimising a deviation determined directly
  from the ratios: $2g/s + s/2g + 3g/b + b/3g + 3s/2b + 2b/3s \ge 6$
  by AM-GM, with equality when the medals achieve the ideal
  proportions.  (This is my preferred approach.)  ($R$)
\end{itemize}

The following table shows when each of these
approaches has failed to predict the boundaries chosen by the Jury.
It will be seen that these algorithms are a less good match before
around 2000, with the Jury apparently having commonly chosen to be
less generous about these choices (fewer gold medals or more bronze
medals) before then.  The boundaries actually chosen are in bold;
other choices of boundaries have the algorithms that would have chosen
them indicated.  Stricter choices of boundaries (fewer gold and silver
medals, then fewer gold medals) are on the left.

\input{gen-gs-alg-table.tex}

\section*{References}

All the code used to implement different algorithms for the above
analyses is available with git at:

\begin{quote}
\texttt{https://git.ukmt.org.uk/medal-boundaries.git} \\
\texttt{git://git.ukmt.org.uk/git/medal-boundaries.git}
\end{quote}

There is also a mirror on GitHub:

\begin{quote}
\texttt{\href{https://github.com/jsm28/medal-boundaries}{https://github.com/jsm28/medal-boundaries}}
\end{quote}

This analysis document may be revised from time to time.  See that
repository for details of previous versions.

The code downloads data from imo-official.  Note that the ideal number
of medals is half the number of non-disqualified contestants (not half
a total number that includes disqualified contestants, in 1991 and
2010), and in 2005 all calculations are done without including two
contestants to whom a translated paper was transmitted inaccurately
(because that was the basis on which boundaries were determined that
year, with those contestants then being given prizes as if they had
scored~7 on the affected problem), meaning that figures differ
slightly from those appearing on imo-official if this is not allowed
for.  Also note that the calculations here were done with data from
before the disqualification in 2016 of an ineligible student from
2010, 2011 and 2012, reflecting the scores available to the Jury at
the time they made their decisions (that student having been given,
before disqualification, scores of 21 in 2010, 14 in 2011 and 18 in
2012).  The code handles those adjustments automatically.

The code is written for simplicity rather than efficiency (for
example, testing all possibilities for gold and silver boundaries
rather than more carefully bounding what cases need to be tested to
find the optimum).

\end{document}
